\usepackage{tikz}

%\usetikzlibrary{shapes.geometric}
\usetikzlibrary{decorations.text}
\usetikzlibrary{quotes}
\usetikzlibrary{calc}
\usetikzlibrary{math}
\usetikzlibrary{intersections}

\newcommand{\ngram}[7]{

\begin{scope}

\newcount\P % number of vertices
\newcount\K 

\newdimen\R
\newdimen\r
\newdimen\w
\newdimen\dr

\tikzmath{
  \P = #1;            % number of vertices
  \Q = #2;            % stellation step
  \C = gcd(\P,\Q);    % number of components
  \N = int(\P/\C);         % number of vertices per component
  \K = min(\Q,\P-\Q); % apparent stellation step
  \c = int(\C-1);     % reduced constants
  \n = int(\N-1);     %   useful as
  \k = int(\K-1);     %   sum indices
  \p = int(\P-1);
  %\n = int(\M-1);     % 
  % Now for some commodities.
  \R = #3;
  \w = #4; \dr = \w/sin(180*(1/2-\K/\P)); \r = \R - \dr;
  function VtoH(\v){
    return int(Mod(Mod(\v*\Q/\C,\N)*\C+(\K<\Q?-1:1)*floor(\v/\N),\P));
  };
  \angle=360/\P;
}



\typeout{Computing vertices.}
\foreach \v   [evaluate = {\h = VtoH(\v)}] in {0,...,\p}
\foreach \rim                              in {-1,0,1} % W, middle, E
{ \coordinate [ name  = \rim h\h
              , alias = \rim v\v
              , at    = (90-\h*\angle : \R-\dr/2+\rim*\dr/2) ];
  %\fill [green] (\rim h\h) circle (1pt) node {\tiny\rim[v\v/h\h]};
  \message{.}; }
\typeout{Done.}

\typeout{Computing sides.}
\foreach \h   [evaluate = {\i = int(Mod(\h+\Q,\P))}] in {0,...,\p}
\foreach \rim                                        in {-1,0,1} % W, m, E
{ \path [name path global = \rim s\h] (\rim h\h)--(\rim h\i);
  %\draw [blue] (\rim h\h) edge ["\tiny\rim s\h", ->] (\rim h\i);
  \message{.}; }
\typeout{Done.}

\typeout{Computing intersections.}
\foreach \v [evaluate = {\h = VtoH(\v)}] in {0,...,\p}
\foreach \j [evaluate={
    \x = 2*\j-(\k+1);
    \y = (\x<0 ? \x : \x+1)*(\K<\Q?-1:1);
    \i = int(Mod(\h+\y,\P));
    \count = int(\v*\k+\j-1);
  }] in {1,...,\k}
\foreach \rimh in {-1,0,1} % w, m, e
\foreach \rimi [evaluate={
    \rimv=\rimh;
    \rimw=int(\x<0?-\rimi:\rimi);
    \name="\rimw \rimh i\count";
  }] in {-1,0,1}
{ % path instead of node: nodes intersections are buggy when rotated
  \path [name intersections = { of = {\rimh s\h} and {\rimi s\i}
                              , by = {\name} }];
  %\fill [red] (\name) circle (1pt) node {\tiny\name};
  \message{.}; }
\typeout{Done.}

\typeout{Weaving the ribbon.}
\draw [#5]
  \foreach \rim in {-1,1}
    \foreach \cn in {0,...,\c}
      \foreach \cv [evaluate={
          \v = \cn*\N+\cv;
          \prec = (\cn*\N+Mod(\cv-1,\N))*\k+\k-1;
        }] in {0,...,\n}
        { (1\rim i\prec)--(\rim v\v)
          \foreach \j [evaluate={
              \i   = \v*\k+\j-1;
              \un  = "--(-1\rim i\i)";
              \der =    "(1\rim i\i)";
            }] in {1,...,\k} {\un\der\pgfextra{\message{.}}} };
\typeout{Done.}

\tikzset{inscribe/.style={draw=none,%
  postaction={decorate,
  decoration={text along path,text/.expanded={#6{0}{##1}},#7},
  decoration/.expanded={#6{1}{##1}}
  }}}

\typeout{Inscribing the ribbon.}
\draw% [red, ultra thick]
    \foreach \cn in {0,...,\c}
      \foreach \cv [evaluate={
          \v = \cn*\N+\cv;
          \w = \cn*\N+Mod(\cv+1,\N);
          \lastins = int(\v*\K+\k);
          \last = "edge[inscribe=\lastins](0v\w)";
        }] in {0,...,\n}
        { (0v\v)
          \foreach \j [evaluate={
              \i   = \v*\k+\j-1;
              \ins = int(\v*\K+\j-1);
              \under  = "edge[inscribe=\ins](-10i\i)(10i\i)";
            }] in {1,...,\k} {\under\pgfextra{\message{.}}}
          \last };
\typeout{Done.}

\end{scope}

}







  %function EulerTotient(\x){
  %  \tot = 0;
  %  for \i in {1,...,\x}{
  %    if gcd(\x,\i) == 1 then {
  %      \tot = \tot+1;
  %    };
  %  };
  %  return \tot;
  %};
  %function ModularPower(\b,\e,\p){
  %  \r = 1;
  %  for \i in {1,...,\e}{
  %    \r = Mod(\r*\b,\p);
  %  };
  %  return int(\r);
  %};
  %function ModularInverse(\b,\p){
  %  return ModularPower(\b,EulerTotient(\p)-1,\p);
  %};
  %\Kinv = ModularInverse(\K,\P);
  %function JumpBy(\h,\i){
  %  %\co  = floor(\v/\N);
  %  %\cv  = Mod(\v,\N);
  %  %\coj = Mod(\i,\C);
  %  %\cvj = floor(\i/\C);
  %  %return \cvj;%Mod(\cv+ModularInverse(\K/\C,\N)*\cvj,\N);
  %  return int(Mod(\h+\i,\P));
  %};
  %function WalkBy(\h,\i){
  %  %\c = floor(\v/\N);
  %  %return int(Mod(\v-\c*\N+\i,\N)+\c*\N);
  %  return int(Mod(\h+\i*\Q,\P));
  %};